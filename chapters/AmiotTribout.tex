\documentclass[output=paper]{LSP/langsci}
\ChapterDOI{10.5281/zenodo.1406993}
\author{Delphine Tribout\affiliation{Université de Lille}\lastand Dany Amiot\affiliation{Université de Lille} }
% \title{Nom et/ou adjectif ? Quelle catégorie d'output pour les suffixés en {-iste} ?} 
\title{\texorpdfstring{Nom et/ou adjectif ? Quelle catégorie d'output pour les suffixés en \textit{-iste} ?}{Nom et/ou adjectif ? Quelle catégorie d'output pour les suffixés en {-iste} ?}}    
\renewcommand{\lsCollectionPaperFooterTitle}{Nom et/ou adjectif ? Quelle catégorie d'output pour les suffixés en \noexpand\textit{-iste} ?}
 
 
% iste should be in emph

\abstract{Cet article aborde la question de la catégorie des construits morphologiques, en particulier le cas des suffixés en \textit{-iste}. Ceux-ci ont la particularité, pour la plupart, d'être ambigus du point de vue de la  catégorie dans la mesure où ils peuvent être noms et/ou adjectifs. Nous montrons qu'il existe deux types de suffixés en \textit{-iste}~: les uns sont fondamentalement des noms, les autres sont fondamentalement des adjectifs, qui peuvent néanmoins être employés comme noms sous certaines conditions. Pour ce dernier cas nous proposons une analyse en termes de \isi{coercion}.}
\maketitle

\begin{document}
\selectlanguage{french}
\il{French|(}
\is{suffixation!in -\emph{iste}|(}

\section{Introduction}
Cet article se focalise sur les catégories construites par la suffixation en \iste. Celle-ci soulève en effet des questions intéressantes car les dérivés qu'elle sert à former semblent appartenir à deux catégories différentes, celles du nom et de l'adjectif.

Les dérivés en \iste\ ont déjà fait l'objet de plusieurs études, notamment par \citep{dubois62, corbin88, roche11}.  Notre étude  se distingue des précédentes dans la mesure où nous nous focalisons ici sur les catégories d'output de la suffixation en \iste. En cela nous adoptons un point de vue différent de celui de \citep{roche11} qui met l'accent sur la sémantique de la suffixation, indépendamment des catégories impliquées. Nous nous intéressons de notre côté aux rapports catégoriels des dérivés en \iste\ qui peuvent souvent être adjectifs et noms. Puisque ces dérivés sont le produit d'une construction morphologique, on peut se demander si une catégorie est  première, construite par la morphologie, et à partir de laquelle serait obtenue l'autre catégorie. Si c'est le cas, se posent alors deux questions~: l'identification de la catégorie première et le mode de formation de l'autre catégorie. On peut au contraire envisager une construction des deux catégories en parallèle, ou encore s'interroger sur une indétermination catégorielle des construits.
C'est à ces questions que nous nous proposons de répondre.

Nous ne mettrons pas en regard, dans cet article, les dérivés en \iste\ avec les dérivés en  \textit{-isme} pour différentes raisons. D'une part, la question des rapports entre les suffixations en \iste\ et en \textit{-isme} a déjà été traitée, notamment par \citep{corbin88} et plus récemment et avec beaucoup de détails par \citep{roche11}. D'autre part, pour la question qui nous intéresse, c'est-à-dire celle des rapports entre catégories adjectivale et nominale des dérivés en \iste, analyser les suffixés en \iste\ comme dérivés ou construits parallèlement aux suffixés en \textit{-isme} ne résout pas le problème. Enfin, il existe un certain nombre de dérivés en \iste\ qui ne présentent aucun correspondant en \textit{-isme}, par exemple \lxm{chimiste, fleuriste, garagiste, pianiste}, ce qui nous semble justifier l'étude des suffixés en \iste\ indépendamment de leur relation avec la suffixation en \textit{-isme}.

Dans un premier temps nous présentons notre méthodologie de constitution du corpus et d'identification des catégories (§~\ref{section:AmiotTribout:methodo}).  Puis nous présentons notre analyse des suffixés en \iste\ (§~\ref{section:AmiotTribout:iste-N-seuls} et \ref{section:AmiotTribout:iste-AN}) et montrons qu'il existe deux cas de figure distincts, tant du point de vue du sens que du point de vue des catégories. Nous montrons que dans le deuxième cas la catégorie adjectivale est première et la catégorie nominale seconde (§~\ref{section:AmiotTribout:orientation}). Pour ce dernier cas, après avoir envisagé deux analyses possibles, l'ellipse et la \isi{conversion}, nous proposons notre propre analyse, en termes de \isi{coercion} (§~\ref{section:AmiotTribout:formation-N}).



\section{Méthodologie} \label{section:AmiotTribout:methodo}
\subsection{Constitution du corpus}\label{section:AmiotTribout:methodo-corpus}
Notre étude des noms et adjectifs suffixés en \iste\ se fonde sur les données de \lexiq\ (\url{http://www.lexique.org/}). Ce lexique comprend 135~000 formes fléchies correspondant à 55~000 lemmes. À chaque forme sont associées différentes informations telles que la catégorie, le genre et le nombre pour les noms et adjectifs, le temps, le mode, la personne et le nombre pour les verbes, la transcription phonétique, etc. En plus des informations morphosyntaxiques, \lexiq\ fournit la fréquence des formes fléchies et des lemmes dans deux corpus, l'un étant un sous-ensemble de textes littéraires récents tirés de Frantext, et l'autre étant un corpus de sous-titres de films.

Pour mener notre étude des noms et adjectifs en \iste, nous avons dans un premier temps extrait de \lexiq\ tous les lemmes se terminant formellement par \iste\ et catégorisés comme noms ou adjectifs, avec leurs fréquences dans les deux corpus. Ces deux fréquences ont été additionnées pour chaque lemme de façon à ne conserver qu'une seule information de fréquence.
Dans un second temps, les noms et adjectifs extraits ont été mis en regard de manière automatique afin d'identifier les noms en \iste\ sans correspondant adjectival, les adjectifs en \iste\ sans correspondant nominal, et les cas de paires nom-adjectif.
Enfin, nous avons validé manuellement les données afin d'écarter les lexèmes se terminant par \textit{-iste} mais qui ne sont pas construits (par exemple \lxm{liste, piste, triste}), ainsi que les lexèmes qui sont bien formés au moyen du suffixe mais dont la suffixation en \iste\ ne correspond pas à la dernière opération morphologique effectuée (par exemple \lxm{chirurgien-dentiste, ex-gauchiste, photojournaliste, néo-communiste}).
Au terme de la validation manuelle notre corpus d'étude contient, selon l'étiquetage de \lexiq~: 277 noms en \iste\ sans adjectif correspondant, 64 adjectifs en \iste\ sans correspondant nominal, et 153 paires nom-adjectif.


Lors de l'examen des données issues de \lexiq\ l'étiquetage catégoriel des formes en \iste\ nous a paru parfois discutable. En effet, parmi les noms sans correspondant adjectival dans la ressource nous avons trouvé plusieurs lexèmes pour lesquels un  adjectif nous semble parfaitement possible et est de surcroît attesté, dans le \textit{TLFi} ou ailleurs. C'est le cas par exemple de \lxm{abstentionniste, carriériste, chauviniste, poujadiste} ou \lxm{utopiste}. À l'inverse, les 64 formes en \iste\ étiquetées comme uniquement adjectivales dans la ressource nous ont semblé pouvoir  également être employées comme des noms. Par exemple des lexèmes tels que \lxm{dualiste, fédéraliste, réformiste} ou \lxm{structuraliste} peuvent avoir un emploi nominal comme le montrent les exemples~(\ref{ex:AmiotTribout:fédéraliste})-(\ref{ex:AmiotTribout:dualiste}) tirés de \textit{Frantext}.

\begin{exe}
\ex \label{ex:AmiotTribout:fédéraliste} il sollicita les ministres en leur confessant sa vieille amitié pour le \textbf{fédéraliste} expiré [\ldots] (Balzac, 1843)
%\ex Le voisin de cachot est un déviationniste, un \textbf{réformiste} (Roy, 1976)
%\ex Cependant les \textbf{réformistes} n'obtenaient rien (Lefebvre 1963)
\ex et les \textbf{réformistes} ne furent pas les moins acharnés à défendre les formules anciennes [\ldots] (Sorel, 1912)
\ex accréditant ainsi pour longtemps, chez les \textbf{structuralistes}, la thèse de l'uni\-vocité [\ldots] (Hagège, 1985)
%\ex l'adresse du chanoine Docre, le fameux \textbf{sataniste} (Huysmans, 1884-1907)
%\ex les \textbf{lettristes} seraient foutus d'égorger les classiques et vice versa (Fallet 1947)
%\ex Depuis que les anarchistes y triomphaient, chassant les évolutionnistes de la première heure, tout craquait (Zola 1885)
\ex \label{ex:AmiotTribout:dualiste} Les résultats que l'on peut obtenir sont très différents de ceux auxquels visaient les \textbf{dualistes} anciens [\ldots]
(David, 1965)

\end{exe}

Nous avons donc eu besoin d'établir des critères afin de déterminer la catégorie des formes en \iste.



\subsection{Critères catégoriels}\label{section:AmiotTribout:criteres-cat}
Si la distinction entre nom prototypique et adjectif prototypique  est clairement établie, il existe néanmoins une zone de flou entre ces deux classes, où les oppositions sont moins tranchées et où la distinction entre catégorie et emploi est plus difficile à établir. Nous présenterons d'abord, très rapidement, les critères des noms et adjectifs prototypiques, puis nous listerons les contextes qui peuvent être ambigus entre les deux catégories.

La grammaire traditionnelle convoque généralement trois critères pour distinguer les catégories nominale et adjectivale~: des critères morphosyntaxiques, sémantiques et syntaxiques (distribution et fonctions). Ces différents critères sont résumés dans le tableau~\ref{tab:AmiotTribout:cat}. 

\begin{table}%[h!]
\caption{Critères catégoriels}
\label{tab:AmiotTribout:cat}
\small
\begin{tabular}{@{ }p{2,7cm}p{4,4cm}p{4,4cm}@{ }}
 \lsptoprule
 & \multicolumn{1}{c}{Nom prototypique} & \multicolumn{1}{c}{Adjectif prototypique} \\
Critère & \multicolumn{1}{c}{(ex. \textit{table})} & \multicolumn{1}{c}{(ex. \textit{grand})} \\
\midrule
morphosyntaxique & genre inhérent, variation en nombre & variation en genre et en nombre \\[+0,2cm]
sémantique & dénote des entités & dénote des propriétés \\[+0,2cm]
distribution & est précédé d'un déterminant, peut être expansé par un adjectif, un syntagme prépositionnel ou une relative & peut être modifié par un adverbe de degré, peut être suivi d'une expansion sous la forme d'un syntagme prépositionnel ou d'une complétive  \\[+0,2cm]
fonctions typiques &  sujet, COD, COI & attribut, épithète\\
\lspbottomrule
\end{tabular}
\end{table}

Relativement opératoires pour distinguer les cas prototypiques, ces critères ont souvent été critiqués (cf. par exemple \cite{wierzbicka98, croft01, croft02, dixon02, haspelmath07} pour ne citer que quelques travaux récents) car ils laissent dans l'ombre de nombreux cas d'usage courant qui enfreignent l'un ou l'autre de ces critères, en particulier les constructions prédicatives~(\ref{ex:AmiotTribout:constr-pred}), que la prédication soit première~(\ref{ex:AmiotTribout:constr-pred-1}) ou seconde~(\ref{ex:AmiotTribout:constr-pred-2}), et l'épithète détachée~(\ref{ex:AmiotTribout:epi}).

\begin{exe}
\ex \label{ex:AmiotTribout:constr-pred}
\begin{xlist}
\ex \label{ex:AmiotTribout:constr-pred-1} Pierre est \{intelligent/avocat\}.
\ex \label{ex:AmiotTribout:constr-pred-2} J'ai un ami \{intelligent/avocat\}.
\end{xlist}

\ex \label{ex:AmiotTribout:epi} Pierre, \{intelligent comme toujours/avocat de renom\}, a signalé que\ldots
\end{exe}

Dans ces constructions, en effet, un nom comme \lxm{avocat} peut s'employer sans déterminant et manifeste ainsi le même comportement qu'un adjectif comme \lxm{intelligent}.  Tous les noms ne peuvent cependant pas entrer dans ce type de constructions~: un  nom comme \lxm{table} ne présente pas la même capacité que \lxm{avocat}, comme le montrent les exemples~(\ref{ex:AmiotTribout:table}).

\begin{exe}
\ex \label{ex:AmiotTribout:table}
\begin{xlist}
\ex[*]{Ceci est table.}
\ex[*]{J'ai un meuble table.}
\ex[*]{Ce meuble, table nouvellement achetée, est vraiment superbe.}
\end{xlist}
\end{exe}

Les noms de profession et de fonction sociale, comme \lxm{avocat}, forment de ce fait une classe spécifique. Ce sont sans doute des noms non prototypiques, mais ils répondent néanmoins à tous les autres critères caractérisant les noms, notamment les critères syntaxiques, distributionnel et fonctionnel. D'autre part, ce comportement caractéristique des noms de profession ou de fonction sociale ne les assimile pas non plus pleinement à des adjectifs~: ils ne peuvent notamment pas être coordonnés avec un adjectif qualificatif comme le montre l'exemple~(\ref{ex:AmiotTribout:intelligent}).

\begin{exe}
\ex[??]{Pierre est grand et avocat.} \label{ex:AmiotTribout:intelligent}
\end{exe}

Nous avons ainsi considéré comme des noms toutes les formes en \iste\ qui  remplissent les critères des noms prototypiques (tableau~\ref{tab:AmiotTribout:cat}), mais aussi celles qui peuvent être employées sans déterminant dans les contextes~(\ref{ex:AmiotTribout:constr-pred}) et (\ref{ex:AmiotTribout:epi}) mais ne peuvent pas être coordonnées avec un adjectif qualificatif comme dans le contexte~(\ref{ex:AmiotTribout:intelligent}).

L'application de ces critères nous a permis d'identifier deux types de formes en \iste~:
celles qui ne sont employées que comme des noms et celles qui sont doublement catégorisées, nom et adjectif.\footnote{Bien que menée dans un cadre radicalement différent, cette distinction en deux sous-ensembles rejoint les deux cas de figure identifiés par \citep{dubois62} et \citep{dubois99}.} Les Sections \ref{section:AmiotTribout:iste-N-seuls} et \ref{section:AmiotTribout:iste-AN} décrivent ces deux cas de figure.




\section{Les formes en \iste\ nominales} \label{section:AmiotTribout:iste-N-seuls}
Les formes en \iste\ ayant un emploi uniquement nominal forment un ensemble relativement homogène  du point de vue morphologique. En effet, ces noms en \iste\ dérivent quasiment tous de noms~(\ref{ex:AmiotTribout:nom-denom}). Deux exemples~(\ref{ex:AmiotTribout:nom-desadj}) sont construits formellement sur des adjectifs mais dérivent en réalité d'unités polylexicales de catégorie nominale~: le criminaliste étudie le droit criminel (sous-type de droit), et l'interniste étudie la médecine interne (sous-type de médecine).

\begin{exe}
\ex \label{ex:AmiotTribout:nom-denom}
\begin{xlist}
\ex \lxm{aubergiste (\orientg auberge), caviste (\orientg cave), dentiste (\orientg dent), machiniste\linebreak (\orientg machine), nouvelliste (\orientg nouvelle), violoniste (\orientg violon)}
\ex \label{ex:AmiotTribout:nom-desadj} \lxm{criminaliste} (\orientg \lxm{droit criminel}), \lxm{interniste} (\orientg \lxm{médecine interne})
\end{xlist}
\end{exe}

Dans quelques cas la base est ambiguë entre nom ou verbe~(\ref{ex:AmiotTribout:nom-deverb}) mais du point de vue du sens une analyse à partir du nom est toujours possible lorsque le nom est associé à une activité.

\begin{exe}
\ex \label{ex:AmiotTribout:nom-deverb} \lxm{archiviste (\orientg archive/archiver), caricaturiste (\orientg caricature/carica\-turer), contorsionniste (\orientg contorsion/contorsionner), copiste (\orientg co\-pie/copier),\linebreak illusionniste (\orientg illusion/illusionner), polémiste (\orientg polé\-mique/polémiquer),\linebreak vocaliste (\orientg vocalise/vocaliser)}\footnote{Dans certains cas la finale du lexème base est tronqué devant le suffixe  \iste, \textit{a fortiori} si elle comprend déjà un [i]. Ainsi pour \lxm{polémiste} le segment final \textit{ique} (si la base est nominale) ou \textit{iquer} (si la base est verbale) est tronqué. Cette troncation n'est pas liée à l'ambiguïté catégorielle de la base~: elle se retrouve également dans \lxm{fataliste}, dérivé de \lxm{fatalité} (ou \lxm{fata\-lisme}). Elle n'est pas davantage spécifique au suffixe \iste\ et s'observe assez fréquemment en français et avec différents suffixes. À ce sujet voir \citep{corbinplenat92}.}
\end{exe}

Enfin, nous avons trouvé un nom formé sur un sigle, \lxm{cibiste} (\orientg \lxm{cb} = \lxm{citizen-band}), et un autre dérivé d'un verbe ou du nom en \textit{-isme} correspondant~: \lxm{exorciste} (\orientg \lxm{exorciser\slash exorcisme)}. 


Du point de vue du sens ces noms sont fondamentalement des noms de métier ou de fonction sociale. Ils correspondent à l'une des deux catégories identifiées par \citep{wolf72}, l'autre étant celle des noms de partisans. De façon générale  ces noms de métiers en \iste\  n'acceptent pas l'emploi adjectival~:

\begin{exe}
\ex[??]{ils sont nombreux à vouloir choisir le \textbf{métier garagiste}} \label{ex:AmiotTribout:garagiste}
\end{exe}

En (\ref{ex:AmiotTribout:garagiste}) \textit{garagiste} ne semble pas fonctionner comme un adjectif en fonction d'épithète dont le rôle serait de qualifier le nom recteur, mais plutôt comme un nom. Il existe en effet une relation d'hypéronymie/hyponymie entre \textit{métier} (l'hypéronyme) et \textit{garagiste} (l'hyponyme).\footnote{Sur les ambiguïtés nom \textit{vs} adjectif en position épithète, cf. par ex. \cite{noailly99}.} Il est cependant possible d'en trouver des exemples, comme en~(\ref{ex:AmiotTribout:metier-adj})~:

\begin{exe}
\ex \label{ex:AmiotTribout:metier-adj}
Je ne suis pas d'un \textbf{tempérament  archiviste} (\textit{Le Monde}, 9 février 2008)\footnote{Exemple emprunté à \citep{rainer16}.}
\end{exe}

Selon \cite{rainer16}, la possibilité d'employer un nom de métier (nom d'agent dans ses termes) en \iste\  comme adjectif qualificatif  dépend de la facilité avec laquelle on peut associer au référent du nom une qualité stéréotypique. Dans le cas de \lxm{archiviste} on peut assez facilement attribuer  au référent la qualité d'être conservateur et ordonné. Cependant, tous les noms de métier en \iste\ n'offrent pas aussi aisément prise aux stéréotypes.
De ce fait, nous ne suivrons pas \cite{rainer16} qui considère qu'il existe, en français actuel, un patron bien établi de formation d'adjectifs en \iste\ par  \isi{conversion} morphologique N~\orientd~A. Une telle analyse ne nous convainc pas dans la mesure où l'emploi d'un nom de métier ou de fonction sociale en position adjectivale ne concerne qu'un petit nombre de noms, et ne semble pas être un processus productif et régulier. 


\section{Les formes en \iste\ doublement catégorisées} \label{section:AmiotTribout:iste-AN}
Les suffixés en \iste\  présentant les deux catégories, nominale et adjectivale, forment en revanche une classe moins homogène du point de vue morphologique. En effet, comme l'a remarqué \cite{roche11}, ils peuvent dériver de noms communs~(\ref{ex:AmiotTribout:deriv-n}), de noms propres~(\ref{ex:AmiotTribout:deriv-npr}), d'adjectifs~(\ref{ex:AmiotTribout:deriv-A}) ou de verbes~(\ref{ex:AmiotTribout:deriv-V}). Ils peuvent également avoir pour base autre chose qu'un lexème, comme des sigles~(\ref{ex:AmiotTribout:deriv-sigles}) ou des syntagmes~(\ref{ex:AmiotTribout:deriv-synt}).

\begin{exe}
\ex
\begin{xlist}
\ex \lxm{anarchiste (\orientg anarchie), centriste (\orientg centre), capitaliste (\orientg capital),\linebreak cycliste (\orientg cycle), gauchiste (\orientg gauche$_{\lxm{n}}$), idéaliste (\orientg idéal$_{\lxm{n}}$), humoriste (\orientg humour),
nombriliste (\orientg nombril), progressiste (\orientg progrès), sexiste (\orientg se\-xe), terroriste (\orientg terreur)} \label{ex:AmiotTribout:deriv-n}
\ex \label{ex:AmiotTribout:deriv-npr} \lxm{bouddhiste (\orientg Bouddha), calviniste (\orientg Calvin), franquiste (\orientg Fran\-co),\linebreak gaulliste (\orientg de Gaulle), marxiste (\orientg Marx), orléaniste (\orientg Orléans), sioniste (\orientg Sion), trotskiste (\orientg Trotsky)}
\ex \label{ex:AmiotTribout:deriv-A} \lxm{communiste (\orientg commun), loyaliste (\orientg loyal), moderniste (\orientg mo\-derne), positiviste (\orientg positif), simpliste (\orientg simple)}
\ex \label{ex:AmiotTribout:deriv-V} \lxm{arriviste (\orientg arriver), conformiste (\orientg se conformer), dirigiste (\orientg di\-riger)}
\ex \label{ex:AmiotTribout:deriv-sigles} \lxm{cégétiste (\orientg cgt), vététiste (\orientg vtt)}
\ex \label{ex:AmiotTribout:deriv-synt} \lxm{fil-de-fériste} (\orientg \textit{fil de fer}), \lxm{jusqu'au-boutiste} (\orientg \textit{jusqu'au bout})
\end{xlist}
\end{exe}

Pour un certain nombre de lexèmes, comme ceux présentés ci-dessous, la base est ambiguë entre nom et adjectif~(\ref{ex:AmiotTribout:base-ambigue}) ou entre verbe et nom~(\ref{ex:AmiotTribout:deriv-V-N}). Selon \cite{roche11}, dans les cas sous~(\ref{ex:AmiotTribout:base-ambigue}) la base formelle (le radical dans les termes de l'auteur) est l'adjectif tandis que le lexème en \iste\ dériverait sémantiquement du nom.

\begin{exe}
\ex \label{ex:AmiotTribout:base-ambigue} \lxm{absentéiste (\orientg absence/absent), existentialiste (\orientg existence/existentiel), féministe (\orientg femme/féminin), individualiste (\orientg individu/individuel), royaliste (\orientg roi/royal)}

\ex \label{ex:AmiotTribout:deriv-V-N} \lxm{alarmiste (\orientg alarmer/alarme), récidiviste (\orientg récidiver/récidive), séparatiste (\orientg séparer/séparation), transformiste (\orientg transformer/trans\-formation)}
\end{exe}

D'un point de vue sémantique  les suffixés en \iste\ doublement catégorisés sont en revanche plus homogènes~: en tant qu'adjectifs ils renvoient à des propriétés comportementales, idéologiques, morales ou philosophiques. En tant que noms ils désignent soit des partisans ou pratiquants d'une idéologie, une philosophie, une discipline ou une activité~(\ref{ex:AmiotTribout:ideo}), soit des habitués d'un certain comportement~(\ref{ex:AmiotTribout:compt}).

\begin{exe}
\ex \label{ex:AmiotTribout:ideo} \lxm{autonomiste, cégétiste, cycliste, gréviste, janséniste, monarchiste}
\ex \label{ex:AmiotTribout:compt}  \lxm{absentéiste, altruiste, fataliste, matérialiste, je-m'en-foutiste}
\end{exe}

\cite{roche11} a également mentionné la possibilité pour les dérivés en \iste\ de désigner des gentilés, comme \lxm{nordiste}, et un cas inclassable, celui de \lxm{unijambiste}, auquel on peut ajouter \lxm{simpliste}. 


Enfin, d'un point de vue syntaxique, ces formes en \iste\ doublement catégorisées semblent se comporter à la fois comme de vrais noms et de vrais adjectifs. Ce sont de vrais adjectifs par les fonctions qu'elles sont capables d'assumer (cf.~critères présentés en~\ref{section:AmiotTribout:criteres-cat}), mais aussi par la capacité qu'elles ont à prendre les marques de degré, comme en~(\ref{ex:AmiotTribout:degre}).

\begin{exe}
\ex \label{ex:AmiotTribout:degre}
\begin{xlist}
%\ex Il a émis des opinions archi-\textbf{socialistes} (Sand, 1843)
%\ex cette profession de foi quasi \textbf{socialiste} (Proudhon, 1846) 
\ex Le jeune Du Camp devient très \textbf{socialiste}. (Flaubert, 1850)
%\ex un discours à tendance très \textbf{socialiste} (Web)
\ex Qui est cet électeur frondeur dans ce territoire fortement \textbf{socialiste}~? (Web)
\ex Fournière ne connaissait pas d'âme plus \textbf{socialiste}  et de cerveau plus fécond que Leroux. (Web)
%\ex Le ministre résident [\ldots] me paraissait peu politique et pas très \textbf{socialiste} (Halimi, 1988) 
\end{xlist}
\end{exe}


En tant que noms, ces formes se comportent également comme de vrais noms~: elles peuvent prendre tout type de déterminant~: défini~(\ref{ex:AmiotTribout:det-def-sg}-b), \mbox{indéfini~(\ref{ex:AmiotTribout:det-indef-sg}-d)}, démonstratif~(\ref{ex:AmiotTribout:det-dem-pl}) ou numéral~(\ref{ex:AmiotTribout:det-num}), et peuvent être employées au singulier comme au pluriel. D'autre part, elles  ne semblent manifester aucune «~déficience catégorielle~» selon les critères de \citep{lauwers14c} et sont pleinement comptables comme le montre la possibilité d'une détermination par \textit{plusieurs} (\ref{ex:AmiotTribout:plusieurs}) ou par un numéral~(\ref{ex:AmiotTribout:det-num}).


\begin{exe}
\ex \label{ex:AmiotTribout:det}
\begin{xlist}
\ex \label{ex:AmiotTribout:det-def-sg} Aux élections, il voterait pour le \textbf{socialiste}. (Aragon 1936)
\ex \label{ex:AmiotTribout:det-def-pl} du côté de la Bastille où les \textbf{socialistes} organisaient un grand rassemblement (Osmont 2012)
\ex \label{ex:AmiotTribout:det-indef-sg} Un \textbf{socialiste} se leva, mais un second extravagant l'arrêta de la main. (Malraux, 1937)
%\ex Des socialistes ont décidé la piteuse opération de Suez (Déon 1960)
%\ex \label{ex:AmiotTribout:det-indef-pl} à l'intérieur du mouvement cohabitaient des \textbf{socialistes} et des anarchistes (Reynaud 1963)
\sloppy
\ex \label{ex:AmiotTribout:plusieurs} plusieurs \textbf{socialistes} de Londres étaient venus nous voir pour dissuader Georges de se marier à l'église (Torrès 1939-1945)
\fussy
%\ex \label{ex:AmiotTribout:det-dem-sg} Pour surnager, ce \textbf{socialiste} se métamorphose (Debray 1996)
\ex \label{ex:AmiotTribout:det-dem-pl} il faudrait qu'il devînt le point de ralliement de ces \textbf{socialistes} (Barrès 1918)
\ex \label{ex:AmiotTribout:det-num} chez les Alsaciens : on avait repéré deux \textbf{socialistes} (Sartre 1949)
\end{xlist}
\end{exe}


En outre, ces formes nominales en \iste\ peuvent, comme n'importe quel nom, être modifiées par un adjectif, un syntagme prépositionnel ou une relative~(\ref{ex:AmiotTribout:modif-N}) et assumer toutes les fonctions nominales~(\ref{ex:AmiotTribout:fonctions-N}).

\begin{exe}
\ex \label{ex:AmiotTribout:modif-N}
\begin{xlist}
\ex le milord Link est détesté de ses collègues pour être partisan de ce \textbf{terroriste} anglais %
(Stendhal, 1835)
\ex  espérons que ce \textbf{réaliste} de profession n'est pas trop romanesque (Sand, 1866)
\ex  les \textbf{fétichistes} qui vénéraient certaines parties de son corps (Duvignaux, 1957)
\end{xlist}
\end{exe}


\begin{exe}
\ex \label{ex:AmiotTribout:fonctions-N}
\begin{xlist}
\ex[S~~~~~~~]{les \textbf{communistes} suscitent l'admiration (Jablonka, 2012)}
\ex[COD~~]{néanmoins il aimait bien les \textbf{communistes} (Osmont, 2012)}
\ex[CdN~~]{au fond de toutes les théories des \textbf{communistes} (Proudhon, 1840)}
\ex[CdA~~]{Celui-là [\ldots] roide comme un \textbf{communiste} (Balzac, 1846)}
\end{xlist}
\end{exe}

Les formes en \iste\ doublement catégorisées semblent donc être autant adjectifs que noms. Par conséquent la question du rapport entre les catégories nominale et adjectivale se pose de manière cruciale. La section suivante est consacrée à l'analyse de cette question.




\section{Orientation catégorielle} \label{section:AmiotTribout:orientation}
Les formes en \iste\ étant morphologiquement dérivées, plusieurs analyses du rapport catégoriel entre adjectif et nom sont possibles~: soit l'une des deux catégories est construite par la suffixation en \iste\ et l'autre est dérivée, et il s'agit alors de déterminer quelle catégorie est première~; soit les deux catégories sont formées en parallèle par la règle de suffixation. \citet[92]{roche11} considère quant à lui que les dérivés en \iste\ sont sous-spécifiés pour la catégorie et que leur emploi nominal ou adjectival est déterminé par le contexte.
Nous ne souscrivons pas à cette analyse par indétermination catégorielle et pensons au contraire que les dérivés en \iste\ sont non seulement catégorisés, mais sont en premier lieu des adjectifs et que leur emploi nominal est second. Pour arriver à ce résultat nous explorons deux critères~: la fréquence des emplois adjectivaux et nominaux~(§~\ref{section:AmiotTribout:frequences}) et l'émergence de ces deux emplois en diachronie~(§~\ref{section:AmiotTribout:diachro}). Nous analysons ensuite les caractéristiques sémantiques des emplois en tant que noms et en tant qu'adjectifs pour montrer l'antériorité de la catégorie adjectivale~(§~\ref{section:AmiotTribout:contraintes-sem}).



\subsection{Fréquences} \label{section:AmiotTribout:frequences}
Afin de déterminer l'orientation de la relation entre deux formes homonymes et de catégories différentes,
 \cite{marchand64} propose de se fonder sur la fréquence d'emploi des deux formes. Selon lui, la forme la plus fréquente est première et  la moins fréquente est dérivée.
Nous avons donc examiné les fréquences adjectivales et nominales des formes en \iste\ doublement catégorisées dans \lexiq.
Ce critère n'a été appliqué qu'aux 153 paires nom-adjectif issues du lexique. Les formes que nous considérons comme doublement catégorisées selon les critères présentés en \ref{section:AmiotTribout:criteres-cat} mais qui sont enregistrées dans \lexiq\ uniquement  comme adjectifs n'ont pas pu être prises en compte, leur fréquence en emploi nominal étant évidemment absente de la ressource.

Nous avons tout d'abord regardé la fréquence moyenne en tant que nom et en tant qu'adjectif pour l'ensemble des 153 formes en \iste\ doublement catégorisées~: elle est de 1.51 pour les formes adjectivales et de 1.73 pour les formes nominales. La différence est minime, d'autant plus qu'une forme joue un rôle perturbateur, \lxm{artiste}, qui a une fréquence en tant que nom de 86.66, alors qu'elle n'est que de 13.2 en tant qu'adjectif.\footnote{L'évolution diachronique de \lxm{artiste} en fait un lexème tout à fait à part dans la série des termes doublement catégorisés.}  Dans la majorité des cas en effet (cf. le tableau \ref{tab:AmiotTribout:freq}, qui regroupe les fréquences des neuf premières formes en \iste\ de notre corpus), une même forme possède une fréquence plus ou moins identique en tant que nom ou en tant qu'adjectif (\lxm{absentéiste, anabaptiste}), sachant que dans certains cas c'est l'emploi nominal qui est un peu plus fréquent (\lxm{activiste}), alors que dans d'autres c'est l'emploi adjectival (\lxm{altruiste}).

\begin{table}
\caption{Fréquence des emplois A et N pour une même forme en \iste}
\label{tab:AmiotTribout:freq}
 \begin{tabular}{lll}
 \lsptoprule
 Lexème & fréq\_A & fréq\_N\\ %table header
  \midrule
abolitionniste  &  0.1 & 0.24 \\
absentéiste & 0.01 & 0.01\\
activiste & 0.75 & 1.42\\
adventiste & 0.08 & 0.27\\
affairiste & 0.15 & 0.28\\
alarmiste & 0.51 & 0.28\\
altruiste & 0.86 & 0.12\\
anabaptiste & 0.42 & 0.34\\
anarchiste & 4.53 & 6.48\\
  \lspbottomrule
 \end{tabular}
\end{table}

Du point de vue des fréquences, rien ne nous permet donc d'affirmer qu'une catégorie serait plus fondamentale que l'autre. 


\subsection{Émergence des catégories en diachronie \textbf{récente}} \label{section:AmiotTribout:diachro}
Nous avons ensuite mené une petite étude en diachronie récente afin de déterminer si les  formes doublement catégorisées avaient un emploi préférentiel de nom ou d'adjectif dans leurs premières attestations. L'hypothèse que nous avons faite est que si une forme en \iste\ possède fondamentalement une catégorie conférée par son mode de formation morphologique, l'autre catégorie devrait être attestée plus tardivement, et son acquisition devrait se faire progressivement. Pour le vérifier, nous avons sélectionné dans le corpus doublement catégorisé huit formes attestées après 1800, soit \lxm{fétichiste} (1824), \lxm{gauchiste} (1839), \lxm{communiste} (1840), \lxm{absentéiste} (1853), \lxm{pacifiste} (1902), \lxm{rousseauiste} (1912), \lxm{centriste} (1922) et \lxm{franquiste} (1936), pour lesquelles nous avons récupéré leurs cent premiers contextes d'apparition dans \textit{Frantext}.

L'analyse des contextes des huit formes étudiées nous a permis de constater que pour chaque forme en \iste, les deux catégories sont attestées quasiment simultanément, comme le montrent les exemples (\ref{ex:AmiotTribout:communiste})-(\ref{ex:AmiotTribout:gauchiste}). Précisons que ces exemples sont pris dans les toutes premières attestations de ces formes relevées dans \textit{Frantext}.


\begin{exe}
\ex \label{ex:AmiotTribout:communiste}
\begin{xlist}
\ex \label{ex:AmiotTribout:communiste-A} la république \textbf{communiste} de Platon suppose [\ldots] (Proudhon, 1840)
\ex \label{ex:AmiotTribout:communiste-N} au fond de toutes les théories des \textbf{communistes} (Proudhon, 1840)
\end{xlist}

\ex \label{ex:AmiotTribout:fetichiste}
\begin{xlist}
\ex  les penseurs théologistes, et même \textbf{fétichistes}, l'appliquèrent mieux (Comte 1852)
\ex la naïve situation des vrais \textbf{fétichistes}. (Comte, 1852)
\end{xlist}

\ex \label{ex:AmiotTribout:gauchiste}
\begin{xlist}
\ex  soutenait un point de vue ultra-\textbf{gauchiste} (Queneau, 1937)
\ex  sa bande de petits \textbf{gauchistes} (Beauvoir, 1951)
\end{xlist}
\end{exe}

En outre, ces formes en \iste\ se comportent pleinement à la fois comme des noms et comme des adjectifs dès les premières attestations. L'analyse des emplois adjectivaux et nominaux en diachronie ne nous permet donc pas davantage que les fréquences de déterminer si une catégorie est antérieure à l'autre. 



\subsection{Contraintes sémantiques}   \label{section:AmiotTribout:contraintes-sem}
Pour finir, nous avons étudié les caractéristiques sémantiques des emplois adjectivaux et nominaux et celles-ci nous conduisent à considérer que la catégorie adjectivale est première et la catégorie nominale seconde.  En effet, nous avons observé que l'emploi nominal est beaucoup plus contraint sémantiquement que l'emploi adjectival. Un adjectif comme \lxm{fantaisiste}, par exemple, peut s'appliquer à différents types de noms~: des noms d'humains (\textit{une personne fantaisiste}) ou d'objets abstraits (\textit{une idée fantaisiste}), et même, bien que plus rarement, des noms d'objets concrets (\textit{un meuble fantaisiste}). Cependant il ne peut être employé comme nom que pour référer à un humain. On pourra dire en effet \textit{un fantaisiste} pour désigner un homme fantaisiste, mais on ne dira jamais, nous semble-t-il, \textit{un fantaisiste} pour parler d'un comportement, ni \textit{une fantaisiste} pour désigner une idée ou une théorie fantaisiste. Ce comportement n'est pas spécifique à \lxm{fantaisiste}, il s'observe au contraire de manière systématique pour tous les adjectifs en \iste~: on peut dire \textit{un \{personnage/projet/bâtiment\} futuriste}, mais \textit{un futuriste} ne peut désigner qu'un homme~; \textit{les \{personnes/thèses\} progressistes} sont tous deux possibles mais \textit{les progressistes} désigne uniquement un groupe d'humains\ldots\ Cette contrainte, très forte, justifie à nos yeux l'antériorité de la catégorie adjectivale et l'orientation adjectif \orientd\ nom.

Se pose alors la question du passage d'adjectif à nom. On peut en effet se demander quel type de procédé permet ce changement de catégorie. La section suivante passe en revue les différentes analyses possibles du phénomène avant de  présenter celle que nous proposons. 


\section{Formation des noms désadjectivaux} \label{section:AmiotTribout:formation-N}
\subsection{Ellipse}
Une première possibilité serait de considérer que les noms en \iste\ issus d'adjectifs sont formés par ellipse, sur le modèle de l'analyse traditionnelle. C'est également le traitement proposé plus récemment par \citet{borerroy10}, \citet{alexiadou13} ou \citet{mcnallydeswart15} dans le cadre d'analyses plus larges concernant les noms désadjectivaux, qu'ils soient ou non suffixés. Selon cette analyse, \textit{un humaniste} serait obtenu à partir de \textit{un homme humaniste} par ellipse du nom \textit{homme}. Une telle analyse pose toutefois plusieurs problèmes.

Le premier problème est celui du nom ellipsé. Dans les cas clairement identifiés comme de l'ellipse, le nom ellipsé varie selon le contexte. Or, dans le cas des noms en \iste\ désadjectivaux, seul un petit nombre de noms pourraient être ellipsés tels que \textit{homme, femme} ou \textit{personne}.

Se pose ensuite la question du genre du nom ellipsé. Lors de l'ellipse d'un nom dans un syntagme nominal, le genre du nom ellipsé est conservé et est visible  sur le déterminant et l'adjectif, comme le montrent les exemples en (\ref{ex:AmiotTribout:ellipse}).

\begin{exe}
\ex \label{ex:AmiotTribout:ellipse}
\begin{xlist}
\ex Il y a plusieurs robes dans la vitrine. J'aime beaucoup la bleue.
\ex À l'animalerie, Paul a choisi une souris grise, et Marie une blanche.
\end{xlist}
\end{exe}
Pour les noms en \iste, l'interprétation qui semble la plus naturelle est `personne qui\ldots '. Or, on ne pourrait expliquer le genre masculin de \textit{un humaniste} si le nom ellipsé était \textit{personne}.

Enfin, l'interprétation des noms en \iste\ pose également problème~: ces noms dénotent systématiquement des humains et ne semblent pas pouvoir désigner un autre type d'objet concret. Or, si les noms en \iste\ étaient obtenus par ellipse, ceux-ci devraient pouvoir dénoter n'importe quel type d'entité, comme dans les exemples en (\ref{ex:AmiotTribout:ellipse}) où \textit{la bleue} désigne un artefact, tandis que \textit{une blanche} dénote un animé.

Il semble donc que l'analyse par ellipse d'un nom ne permette pas d'expliquer la formation de ces noms  en \iste\ issus d'adjectifs.


\subsection{Conversion}
Une autre possibilité est d'analyser  ces noms comme des converts. En effet, la  \isi{conversion} adjectif~\orientd~nom existe en français \citep{Corbin87, kerleroux96} comme dans le cas des exemples en (\ref{ex:AmiotTribout:conv}).
\begin{exe}
\ex \label{ex:AmiotTribout:conv} \lxm{calme$_{\textsc{a}}$\orientd calme$_{\textsc{n}}$, ~bleu$_{\textsc{a}}$\orientd bleu$_{\textsc{n}}$}
\end{exe}

\cite{corbin88} analyse d'ailleurs les noms en \iste\ comme des convertis à partir d'adjectifs. Cette analyse se justifie dans la mesure où les noms en \iste\ montrent toutes les propriétés des noms, comme cela a été présenté en Section \ref{section:AmiotTribout:iste-AN}. Toutefois, ils manifestent aussi des propriétés adjectivales, notamment la possibilité d'être modifiés par un adverbe de degré, comme le montrent les exemples en (\ref{ex:AmiotTribout:proprietes-adj}) trouvés sur le Web.

\begin{exe}
\ex \label{ex:AmiotTribout:proprietes-adj}
\begin{xlist}
\ex Il reste que Letizia d'Espagne, critiquée par \textbf{les très royalistes} et admirée par les moins conventionnels, incarne la princesse moderne par excellence
\ex \textbf{les très idéalistes} ne se retrouvent pas facilement ensemble et au contraire se trouvent souvent en plein contentieux
\ex Seuls les esprits étriqués ont jamais pensé que le réel se limitait à ce que nous en percevions ! clament \textbf{les plus idéalistes}.
\ex En tête de liste, l'enseignement. \textbf{Les plus alarmistes} pourraient imaginer des professeurs purement et bonnement remplacés par des ordinateurs
\end{xlist}
\end{exe}


Or, un nom ne peut normalement pas être modifié par un adverbe, sauf s'il est coercé par une construction prédicative \citep{Lauwers14} comme \textit{femme} dans \textit{Marie fait très femme maintenant}, qui sera discuté dans la section suivante (exemple (\ref{ex:AmiotTribout:override})).
Cette faculté à être modifiés par un adverbe montre que les noms de partisans en \iste\ ne sont pas des noms ordinaires. De ce fait, une analyse par  \isi{conversion} ne nous paraît pas satisfaisante car elle ne saurait expliquer cette faculté. En effet, un convert présente toutes les propriétés de la catégorie à laquelle il appartient, comme l'a souligné \cite{kerleroux96}, mais ne présente normalement pas les propriétés syntaxiques de sa base. C'est pourquoi nous présentons dans la section suivante une analyse alternative. 



\subsection{Coercion} \label{section:AmiotTribout:coercion}\is{coercion}
Pour rendre compte des propriétés à la fois adjectivales et nominales des noms de partisans en \iste\ nous proposons une analyse par \isi{coercion}. Pour cela nous présentons d'abord les différents types de \isi{coercion} (§~\ref{section:AmiotTribout:types-coercion}) avant de montrer en quoi l'«~override \isi{coercion}~» permet de rendre compte des particularités du passage d'adjectif à nom qui résistaient aux analyses par ellipse ou par  \isi{conversion} (§~\ref{section:AmiotTribout:override})\footnote{Nous choisissons de  traduire le terme \textit{override} par \textit{forçage}, c'est en effet le terme qui nous a semblé le mieux correspondre à la définition donnée~; cf. \textit{infra}.}. Précisons que cette analyse par \isi{coercion} est similaire à celle proposée par \cite{lauwers08, lauwers14b} pour les noms de propriété désadjectivaux.

\subsubsection{Différents types de \isi{coercion}}
\label{section:AmiotTribout:types-coercion}
\largerpage

Depuis les années 1990, une abondante littérature a été consacrée à la \isi{coercion}. On peut se reporter par exemple à \citep{pustejovsky91, jackendoff97, michaelis03, francismichaelis03, lauwerswillems11}. Comme l'ont établi \citep[1219]{lauwerswillems11} «~at the basis of \isi{coercion}, there is a mismatch %
%(cf.~Francis and Michaelis 2003) 
\citep[cf.][]{francismichaelis03} %
%Francis-Michaelis
%
between the semantic properties of a selector (be it a construction, a word class, a temporal or aspectual marker) and the inherent semantic properties of a selected element, the latter being not expected in that particular context.~».

\cite{audringbooij16} distinguent trois types de \isi{coercion}~: la \isi{coercion} par sélection, la \isi{coercion} par enrichissement et la \isi{coercion} par forçage.
Les deux premiers types sont fondamentalement des adaptations contextuelles de traits sémantiques~; la \isi{coercion} par forçage quant à elle, qui est le type de \isi{coercion} le plus fort et celui qui possède la portée la plus large, est fondée sur l'«~override principle~» de \citep[9]{michaelis03}~: «~\textit{Override principle}. If lexical and structural meanings conflict, the semantic specifications of the lexical element conform to those of the grammatical structure with which that lexical item is combined.~».
Dans la \isi{coercion} par forçage en effet, c'est le contexte qui prend le pas sur les propriétés (sémantiques, catégorielles ou syntaxiques) de l'item coercé et lui impose son interprétation.


En français, F. Kerleroux, dès le début des années 1990 (cf. notamment \citealt{kerleroux91, kerleroux96}), a proposé une analyse relativement similaire par le biais de la notion de «~distorsion catégorielle\is{categorial distortion}~». En s'appuyant sur l'opposition opérée par \cite{milner89} entre terme et position, elle a en effet rendu compte de cas comme celui de l'exemple (\ref{ex:AmiotTribout:elegant}) où l'adjectif \textit{élégant} est utilisé en position nominale.

\begin{exe}
\ex \label{ex:AmiotTribout:elegant}   Il est d'un élégant !
\end{exe}

Pour elle, c'est l'inadéquation entre la catégorie du terme lui-même (un adjectif) et la position dans laquelle il est employé (dans un syntagme nominal après un déterminant) qui rend compte du comportement et de l'interprétation particulière de \textit{élégant} dans ce contexte. Une telle analyse correspond aussi plus ou moins à celle que propose \cite{lauwers14b} pour certains noms abstraits désadjectivaux.


\subsubsection{La \isi{coercion} par forçage (\textit{override \isi{coercion}})} \label{section:AmiotTribout:override}
\largerpage
L'analyse en termes de \isi{coercion} est fréquente en Grammaire de Construction\is{Construction Grammar} pour rendre compte de cas comme celui sous (\ref{ex:AmiotTribout:override})~:

\begin{exe}
\ex \label{ex:AmiotTribout:override} Marie fait très femme
\end{exe}


Dans cet exemple, un nom (\textit{femme}) est employé en contexte typiquement adjectival, c'est-à-dire un contexte prédicatif, avec modification par l'adverbe d'intensité \textit{très} (cf. §~\ref{section:AmiotTribout:criteres-cat}). \textit{femme} ne devient pas réellement un adjectif, mais son interprétation, dans un contexte comme celui-ci, va être semblable à celle d'un adjectif~: ce qui importe ici, ce sont les propriétés qui lui sont prototypiquement associées.

Une telle analyse peut être facilement transposées aux adjectifs en \iste\ employés comme noms. Nous faisons donc l'hypothèse que ces adjectifs sont coercés en étant intégrés à un syntagme nominal (SN), c'est-à-dire un contexte fait pour être saturé par un nom~:

\begin{exe}
\ex
\begin{xlist}
\ex \label{ex:AmiotTribout:sn-proto} cas prototypique~: [$_{SNcompt}$ Dét N]  $\longleftrightarrow$ `SN comptable'
\ex \label{ex:AmiotTribout:sn-coerce}  \isi{coercion} par forçage~: [$_{SNcompt}$ Dét A]  $\longleftrightarrow$ `SN comptable'
\end{xlist}
\end{exe}

\clearpage 
La représentation, très simplifiée, emprunte aux Grammaires de Construction\is{Construction Grammar} (notamment \citep{Booij10}), pour lesquelles une construction, par exemple un SN, est une association forme/sens~: à gauche de la double flèche, entre crochets droits, figure la forme, alors qu'à la droite, encadré par des guillemets simples, figure le sens. En (\ref{ex:AmiotTribout:sn-coerce}), le fait de placer un adjectif dans une place normalement dévolue à un nom (cf. le cas prototypique illustré par (\ref{ex:AmiotTribout:sn-proto})) confère donc à l'ensemble une interprétation nominale, identique à celle qu'elle aurait si le terme était un nom.  Nous avons choisi de préciser que le SN est un SN comptable pour justifier de la sémantique des N en \iste\ --ils dénotent des individus--, et pour justifier de la possibilité qu'ils ont d'être précédés de tous types de déterminants (cf. les ex.~(\ref{ex:AmiotTribout:det})).\footnote{Pour rendre compte de la formation et des caractéristiques des noms de propriété désadjectivaux, \cite{lauwers14b} avait quant à lui fait l'hypothèse que les adjectifs étaient intégrés à des SN massifs.}

Avant de mentionner les avantages d'une telle analyse, nous voudrions revenir sur un point, qui concerne leur éventuelle lexicalisation. Le procédé de \isi{coercion} tel que nous venons de le décrire explique l'apparition des formes nominales issues des adjectifs correspondants. Certaines formes nominales peuvent cependant être utilisées à une fréquence importante et être consacrées par l'usage.\footnote{Sur le rôle et la fonction de la fréquence, voir par exemple \cite{bybee06, bybeethompson97, ellis02, gries13}.} Il en est ainsi par exemple de \textit{communiste}, dont la fréquence d'emploi en tant que nom  (37.28 si on se base sur \lexiq) est sensiblement identique à celle qu'il a en tant  qu'adjectif (36.17 dans \lexiq). Certaines formes nominales sont même devenues nettement plus fréquentes que les formes adjectivales correspondantes, c'est le cas de \textit{terroriste} (N\verb!:!19.12 \textit{vs} A\verb!:!8.87). De telles formes peuvent alors être lexicalisées en tant que noms, et figurer à ce titre dans des dictionnaires. \textit{Optimiste} (A\verb!:!8.4 \textit{vs} N\verb!:!1.84), \textit{réaliste} (A\verb!:!12.86 \textit{vs}  N\verb!:!1.02), \textit{intimiste}  (A\verb!:!0.24 \textit{vs} N\verb!:!0.07) ou \textit{fantaisiste} (A\verb!:!2.56 \textit{vs} N\verb!:!0.97) restent en revanche assez fondamentalement associées à la catégorie de l'adjectif, et la \isi{coercion} joue sans doute encore pleinement son rôle lorsqu'ils sont employés en tant que noms.

  
Cette analyse par \isi{coercion} possède, selon nous, au moins deux grands avantages~:
\begin{itemize}
\item[(i)] d'une part elle permet d'expliquer la facilité avec laquelle il est possible d'obtenir ces formes désadjectivales~: la \isi{coercion} étant un phénomène syntaxique, cela rend compte du caractère systématique de leur formation~;
\item[(ii)] d'autre part elle permet aussi, et surtout, d'expliquer pourquoi les noms désadjectivaux en \iste\ peuvent encore avoir des propriétés adjectivales, notamment être modifiés par un adverbe de degré (par ex. \textit{les très idéalistes} en (\ref{ex:AmiotTribout:proprietes-adj}))~: en tant que formes coercées, les désadjectivaux en \iste\  ne sont pas pleinement des noms mais des adjectifs en emploi nominal.
\end{itemize}

 
Cette analyse des noms issus d'adjectifs en \iste\  s'intègre à une analyse plus large de l'alternance adjectif/nom, \rephrase{}{un} %sinon "d'un" plus loin va à la marge
phénomène présent dans l'ensemble du lexique et que nous avons décrit dans \citeauthor{amiottribout17} (à paraître)~: n'importe quel adjectif, qu'il soit  simple (\lxm{jeune, grand}), morphologiquement construit (\lxm{ambitieux, parlementaire}) ou issu d'un participe (\lxm{blessé, perdant}) peut être employé com\-me nom pour désigner un humain à condition que la propriété dénotée par l'adjectif soit susceptible de caractériser l'humain. L'ambition, par exemple, peut caractériser une personne (ex. \textit{un homme ambitieux}) c'est pourquoi l'adjectif \lxm{ambitieux} peut être utilisé comme nom pour référer à un être humain (\textit{un ambitieux}).
À l'inverse, un adjectif comme \lxm{argileux} semble difficilement pouvoir caractériser un être humain et ne peut donc être employé comme nom d'humain (\textit{??un argileux}).
 

Par rapport à ce cas général, la spécificité de la suffixation par \iste\ réside dans ses affinités particulières avec l'humain~:  en témoigne le sémantisme des dérivés nominaux, qui dénotent des noms de métier et de fonction sociale (par ex. \lxm{dentiste, garagiste})~; en témoigne aussi le sémantisme des dérivés adjectivaux, qui dénotent généralement des propriétés relatives à des comportements (\lxm{absentéiste, alarmiste, individualiste}), des croyances (\lxm{bouddhiste, calviniste, janséniste}), des idéologies (\lxm{marxiste, capitaliste, progressiste}) etc., c'est-à-dire des propriétés qui sont toutes aptes à caractériser, directement ou indirectement, l'humain. C'est la raison pour laquelle tous les adjectifs en \iste\ sont propres  à l'emploi nominal à référence humaine, contrairement à d'autres types de suffixations, comme  --\textit{eux} ou --\textit{aire}, dont les dérivés ne possèdent pas tous cette capacité (par ex. \lxm{argileux, budgétaire}).


\section{Conclusion}
Dans cet article nous nous sommes intéressées à la suffixation en \iste\  \rephrase{car}{puisque} ce procédé de formation de lexèmes soulève des questions peu étudiées jusqu'à présent et qui concernent la relation entre lexèmes construits de forme identique mais de catégories différentes, ici adjectivales et nominales.

  
Nous avons montré qu'il existe deux types de suffixés en \iste ~:

\begin{enumerate}
\item[(i)] les formes nominales auxquelles ne correspondent pas d'adjectif. Ces formes sont sémantiquement très homogènes, elles dénotent des métiers ou fonctions sociales. Même si l'emploi adjectival n'est pas totalement exclu pour certains de ces noms, celui-ci reste assez souvent ambigu (apposition ou adjectif épithète~?).

\item[(ii)] les formes adjectivales auxquelles correspondent des noms. Celles-ci présentent elles aussi une grande homogénéité d'un point de vue sémantique~: en tant qu'adjectifs elles renvoient à des propriétés comportementales, idéologiques, mo\-rales ou philosophiques~; en tant que noms elles dénotent les partisans ou les pratiquants d'une idéologie, une philosophie, une discipline, une activité, ainsi que les habitués d'un certain comportement. Pour ces noms, nous n'avons pas repris l'analyse de \cite{roche11}, à savoir la sous-spécification entre catégories adjectivale et nominale. Nous considérons quant à nous que ces noms, ayant nécessairement une référence humaine, sont issus de la forme adjectivale corres\-pondante. En outre, nous avons montré que ces noms, qui présentent toujours des propriétés adjectivales, sont obtenus par \isi{coercion}.
\end{enumerate}
Ce traitement par \isi{coercion} des noms issus d'adjectifs en \iste\  s'intègre à une analyse plus large d'un phénomène observé dans tout le lexique, à savoir que tout adjectif  est employable comme nom pour désigner un humain si la propriété qu'il dénote peut caractériser l'humain (\citeauthor{amiottribout17}, à paraître). Par ailleurs,​ les noms abstraits issus d'adjectifs homonymes tels que \textit{le beau, l'utile, l'humanitaire}\ldots\ ont été traités par \cite{lauwers08, lauwers14b} comme des adjectifs coercés dans des emplois nominaux. Notre analyse des noms d'humains s'articule donc parfaitement avec celle de Lauwers et vient ainsi compléter la description des noms homonymes d'adjectifs en français. Enfin, il existe également des noms d'objets obtenus à partir d'adjectifs homonymes tels que \lxm{commode, collant} ou \lxm{bleu}. Ils diffèrent toutefois des noms d'humains sur deux points~: i) ils ne peuvent pas être modifiés par un adverbe~; ii) l'emploi nominal pour désigner des artefacts n'est pas aussi systématique que pour désigner des humains. Ces noms d'artefacts restent à étudier afin de déterminer comment leur description s'articule avec celle que nous avons proposée pour les noms d'humains, ainsi qu'avec celle proposée par Lauwers pour les noms abstraits.

\is{suffixation!in -\emph{iste}|)}
\il{French|)}

{\sloppy
    \printbibliography[heading=subbibliography,notkeyword=this]
}


\end{document}
