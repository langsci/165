%add all your local new commands to this file

\makeatletter
\def\blx@maxline{77}
\makeatother
\newcommand{\lda}{\ensuremath{\lambda\;}}
\newcommand{\la}[1]{\ensuremath{\lambda #1 \;}}
\newcommand{\lx}{\ensuremath{\lambda x \;}}
\newcommand{\ly}{\ensuremath{\lambda y \;}}
\newcommand{\E}[1]{\ensuremath{\exists #1 \;}}
\newcommand{\f}[2]{\ensuremath{\text{\ttfamily #1}(#2)}}
\newcommand{\et}{\ensuremath{\wedge}}
\newcommand{\ou}{\ensuremath{\vee}}
\newcommand{\lxm}[1]{\textsc{#1}}


\avmoptions{center}
\avmfont{\upshape\scshape}
\avmsortfont{\upshape\itshape}
\avmvalfont{\upshape\itshape}

	
%% Citations
\newcommand{\citepos}[1]{\citeauthor{#1}'s (\citeyear{#1})}
\newcommand{\citeposp}[2]{\citeauthor{#1}'s (\citeyear{#1}: #2)}

\renewcommand\multicitedelim{\addcomma\space}

%%https://tex.stackexchange.com/questions/125690/bibtex-and-genitive-possessive-s-the-proper-way-to-obtain-kurans-1989-mo/125706
%\makeatletter
%
%% make numeric styles use name format
%\patchcmd{\NAT@test}{\else \NAT@nm}{\else \NAT@nmfmt{\NAT@nm}}{}{}
%
%% define \citepos just like \citet
%\DeclareRobustCommand\citepos
%  {\begingroup
%   \let\NAT@nmfmt\NAT@posfmt% ...except with a different name format
%   \NAT@swafalse\let\NAT@ctype\z@\NAT@partrue
%   \@ifstar{\NAT@fulltrue\NAT@citetp}{\NAT@fullfalse\NAT@citetp}}
%
%\let\NAT@orig@nmfmt\NAT@nmfmt
%\def\NAT@posfmt#1{\NAT@orig@nmfmt{#1's}}
%
%\makeatother


%% Sacha
\makeatletter
\def\maxwidth#1{\ifdim\Gin@nat@width>#1 #1\else\Gin@nat@width\fi}
\makeatother

\newcommand{\movedfootnote}[1]{\textcolor{black}{\footnote{\textcolor{black}{#1}}}}
\renewcommand{\marginpar}[1]{}

%% Strnadova

\newcommand{\lgc}{\cellcolor{lightgray}}
%lexème
%suffixe
\newcommand{\sx}{-\emph}
%relation dérivationnelle orientée
\newcommand{\der}{ $\rightarrow$ }
%relation
% \renewcommand{\rel}{ $\sim$ }
\newcommand{\rel}{ $\sim$ }
%exemple phrase
\newcommand{\exph}[1]{\emph{#1}}

%% Amiot  Tribout
\newcommand{\iste}{\emph{-iste}}
\newcommand{\lexiq}{\emph{Lexique~3}}
\newcommand{\orientg}{<}
\newcommand{\orientd}{$>$}

%% Spencer

\newcommand{\emphterm}[1]{\textbf{#1}}     %for any in-text cited expression
\newcommand{\textex}[1]{\emph{#1}}     %for any in-text cited expression
\newcommand{\page}[1]{\mbox{(p. #1)}}
\newcommand{\hpsgtag}[1]{\raisebox{0.2ex}{{\tiny\fbox{#1}}}}
\newcommand{\gap}[1]{\rule{{#1}em}{0.5pt}}
\newcommand{\wf}[1]{\emph{#1}}	%for any cited linguistic example
\newcommand{\af}[1]{\emph{#1}}	%for any affix(-like) element

%% Hatout Namer

%
% \def\expo#1{%
%   \raisebox{.55ex}{\scriptsize #1\kern+.15em}}
% \def\indi#1{%
%   \raisebox{-.55ex}{\scriptsize #1\kern+.15em}}


\newcommand{\indi}[1]{\textsubscript{\scriptsize #1}}
\newcommand{\expo}[1]{\textsuperscript{\scriptsize #1}}
%\newcommand{\ieme}{\textsuperscript{e}}
\newcommand{\itemarr}{Item et Arrangement}
\newcommand{\paradis}{\protect\textsf{ParaDis}}

%% Boyé


\newcommand{\textipan}[1]{\textsf{\textipa{#1}}}
\newcommand{\lplus}{\ensuremath{\oplus}}
\newcommand{\lmoins}{\ensuremath{\ominus}}
\newcommand{\lpause}{\ensuremath{\oslash}}
\newcommand{\fldr}{\ensuremath{\longrightarrow}}
\newcommand{\refp}[1]{(\ref{#1})}
\newcommand{\trad}[1]{`#1'}
\newcounter{nex}
\setcounter{nex}{1}
\newcommand{\nex}[1][\thenex]{\setcounter{nex}{#1}\thenex\stepcounter{nex}}
\newcommand{\numsep}[1]{\num[group-separator={,}]{#1}}
\renewcommand{\numsep}[1]{\numprint{#1}}
\newcommand{\tradnum}[1]{\trad{\numsep{#1}}}
% has strange side effects
\tikzset{every tree node/.style={align=center, anchor=north}}
\tikzset{every roof node/.append style={inner sep=0.1pt,text height=2ex,text depth=0.3ex}}


\npstyleenglish

% CruzStump

%\newcolumntype{L}[1]{>{\raggedright\let\newline\\\arraybackslash\hspace{0pt}}m{#1}}
%\newcolumntype{C}[1]{>{\centering\let\newline\\\arraybackslash\hspace{0pt}}m{#1}}
%\newcolumntype{R}[1]{>{\raggedleft\let\newline\\\arraybackslash\hspace{0pt}}m{#1}}
\def\checkmark{\tikz\fill[scale=0.4](0,.35) -- (.25,0) -- (1,.7) -- (.25,.15) -- cycle;}

% BascianoMelloni

\newcommand{\tld}{\textasciitilde{}}

%% Side by side gloses separated by arrow
\newcommand{\sbsgll}[6]{\begin{tabular}[t]{p{0.25\textwidth}m{0.1\textwidth}p{0.4\textwidth}}
#1 & \textrightarrow & #2 \\
#3 & & #4 \\
#5 & & #6 \\ % translation
\end{tabular}}

\newcommand{\sbsglll}[8]{\begin{tabular}[t]{p{0.25\textwidth}m{0.1\textwidth}p{0.4\textwidth}}
#1 & \textrightarrow & #2 \\
#3 & & #4 \\
#5 & & #6 \\
#7 & & #8 \\ % translation
\end{tabular}}

\makeatletter
\newcommand{\nolistbreak}{%
  \let\oldpar\par\def\par{\oldpar\nobreak}% Any \par issues a \nobreak
  \@nobreaktrue% Don't break with first \item
}
\makeatletter

 %%%%%%%%%%%%%%%%%%%%%%%%%%%%%%%%%%%%%%%%%%
%  Fix for markup in titles 
\renewcommand{\lsCollectionPaperFooterTitle}{\@title}
 \renewcommand{\includepaper}[1]{
	\begin{collectionpaper}
	\begin{refsection} 
	\DeclareCiteCommand{\fullciteFooter}
		{\defcounter{maxnames}{\blx@maxbibnames}%
		  \usebibmacro{prenote}}
		{\usedriver
		   {\DeclareNameAlias{sortname}{default}}
		   {\thefield{entrytype}}}
		{\multicitedelim}
		{\usebibmacro{postnote}}
	\renewcommand{\lsCollectionPaperCitationText}{\fullciteFooter{#1}}  
% 	\renewcommand{\lsCollectionPaperCitationText}{suspended in localcommands.tex}  
	\include{#1}%
  \addtocounter{page}{-1}
	\edef\lsCollectionPaperLastPage{\thepage}	 
  \addtocounter{page}{1} 
	\onlyAuthor		
	\renewcommand{\newlineCover}{}
	\renewcommand{\newlineSpine}{}
	\renewcommand{\newlineTOC}{}
	\StrSubstitute{\@author}{,}{ and }[\authorTemp]
	\StrSubstitute{\authorTemp}{\&}{ and }[\authorTemp]
	\StrSubstitute{\lsCollectionEditor}{,}{ and }[\editorTemp]
	\StrSubstitute{\editorTemp}{\&}{ and }[\editorTemp]  
	\immediate\write\tempfile{@incollection{#1,author={\authorTemp},title={{\lsCollectionPaperFooterTitle}},booktitle={{\lsCollectionTitle}},editor={\editorTemp},publisher={Language Science Press.},Address={Berlin},year={\,\lsYear},pages={\lsCollectionPaperFirstPage --\lsCollectionPaperLastPage},doi={\lsChapterDOI},options={skipbib=true,skiplab=true}}}
	\end{refsection}
	\end{collectionpaper}}
 %%%%%%%%%%%%%%%%%%%%%%%%%%%%%%%%%%%%%%%%%%




%\renewcommand{\rephrase}[2]{{\color{red!80!black}#2}\marginpar{replaced `#1'}}
\renewcommand{\rephrase}[2]{#2}


\newcommand{\fnref}[1]{Footnote~\ref{#1}}


\frenchbsetup{SmallCapsFigTabCaptions=false}
\addto\captionsfrench{\def\tablename{Tableau}}

\renewcommand{\sectref}[1]{Section~\ref{#1}}